\documentclass[reprint,amsmath,amssymb,aps]{revtex4-2}

\usepackage{graphicx}% Include figure files
\usepackage{dcolumn}% Align table columns on decimal point
\usepackage{bm}% bold math
\usepackage{hyperref}% add hypertext capabilities
\usepackage[spanish,mexico]{babel}
\usepackage{float}%usar [H]
\usepackage{diagbox}
% \graphicspath{{imagenes/}}%Carpetas donde estaran las imagenes

\begin{document}

\preprint{APS/123-QED}

\title{Diseño de una Memoria RRAM 2T2R no Volátil}
\author{José de Jesús de la Rosa de la Rosa}
\email{a228835@alumnos.uaslp.mx}
\affiliation{Facultad de Ciencias, Universidad Autónoma de San Luis Potosí.}
\date{22 de junio de 2022}

\begin{abstract}
Diferentes tecnologías en escalas nanométricas han llevado el diseño de arquitecturas de memoria a un sistema híbrido con lógica CMOS, para lograr una alta densidad de dispositivos con un bajo consumo de energía.
\end{abstract}

%\keywords{HOLA,JH,OK}

\maketitle 

\section{Introducción}
Las memorias no volátiles (NVM) muestran características como alta densidad, bajo costo, rápido acceso para lectura y escritura (R/W), bajo consumo energético, alto desempeño, alta resistencia y retención. En la actualidad los dispositivos de memoria basados en Flash \cite{manem, maju}, representan la tecnología mas común entre los dispositivos NVM (memoria no volátil) debido a su alta densidad y bajos costos de fabricación. Sin embargo, la tecnología Flash sufre de baja resistencia, baja velocidad de escritura y altos voltajes para el proceso de escritura \cite{wang}. 

% \begin{figure}[H]
% 	\centering
% 	\includegraphics[width=0.48\textwidth]{imagen}
% 	\caption{Clasificación de los efectos de conmutación resistiva que se consideran para aplicaciones de memoria no volátil \cite{waser2009redox}.}
% 	\label{imagen}
% \end{figure}


\section{Conclusiones}
A pesar de las diferentes arquitecturas de memorias basadas en RRAM reportadas en la literatura, no ha sido reportada una celda que logre una implementación funcional debido los problemas derivados de la variabilidad de los dispositivos.
  
\bibliographystyle{unsrtnat}
\nocite{*}
\bibliography{citas}% Produces the bibliography via BibTeX.

\end{document}
